\documentclass[fleqn, a4paper, 11pt, oneside]{amsart}
%\usepackage[top = 2cm, bottom = 1cm, left = 1cm, right = 1cm]{geometry}
\usepackage{exsheets, tasks}
\usepackage{amsmath, amssymb, amsthm} %standard AMS packages
\usepackage{marginnote} %marginnotes
\usepackage{gensymb} %miscellaneous symbols
\usepackage{commath} %differential symbols
\usepackage{xcolor} %colours
\usepackage{cancel} %cancelling terms
\usepackage[free-standing-units, space-before-unit]{siunitx} %formatting units
\usepackage{tikz, pgfplots} %diagrams
\usetikzlibrary{calc, hobby, patterns, intersections, decorations.markings}
\usepackage{graphicx} %inserting graphics
\usepackage{hyperref} %hyperlinks
\usepackage{datetime} %date and time
\usepackage{ulem} %underline for \emph{}
\usepackage{xfrac} %inline fractions
\usepackage{enumerate,enumitem} %numbered lists
\usepackage{float} %inserting floats
\usepackage{circuitikz}[american voltages, american currents] %circuit diagrams
\usepackage[utf8]{inputenc}

\newcommand\numberthis{\addtocounter{equation}{1}\tag{\theequation}} %adds numbers to specific equations in non-numbered list of equations

\newcommand{\AxisRotator}[1][rotate=0]{
	\tikz [x=0.25cm,y=0.60cm,line width=.2ex,-stealth,#1] \draw (0,0) arc (-150:150:1 and 1);%
} %rotation symbols on axes

\theoremstyle{definition}
\newtheorem{example}{Example}
\newtheorem{definition}{Definition}

\theoremstyle{theorem}
\newtheorem{theorem}{Theorem}

\newcommand{\curl}{\mathrm{curl\,}}

\makeatletter
\@addtoreset{section}{part} %resets section numbers in new part
\makeatother

\renewcommand{\thesubsection}{(\arabic{subsection})}
\renewcommand{\thesection}{(\arabic{section})}

\renewcommand{\emph}{\uline}

\renewcommand{\tilde}{\widetilde}

%section headings on left
\makeatletter
\def\specialsection{\@startsection{section}{1}%
	\z@{\linespacing\@plus\linespacing}{.5\linespacing}%
	%  {\normalfont\centering}}% DELETED
	{\normalfont}}% NEW
\def\section{\@startsection{section}{1}%
	\z@{.7\linespacing\@plus\linespacing}{.5\linespacing}%
	%  {\normalfont\scshape\centering}}% DELETED
	{\normalfont\scshape}}% NEW
\makeatother

%forces newline after subsection
\makeatletter
\def\subsection{\@startsection{subsection}{3}%
	\z@{.5\linespacing\@plus.7\linespacing}{.1\linespacing}%
	{\normalfont\itshape}}
\makeatother

\settasks{counter-format = tsk[1].}

\SetupExSheets{solution/print = true}

%opening
\title{Quantum and Solid State Physics : Assignment 5}
\author
{
	Aakash Jog\\
	ID : 989323563
}
\date{\formatdate{19}{11}{2015}}

\begin{document}

\tikzset{->-/.style={decoration={
  markings,
  mark=at position #1 with {\arrow{>}}},postaction={decorate}}}

\maketitle
%\setlength{\mathindent}{0pt}

\setcounter{question}{1}
\begin{question}
	Consider a particle described by a wave function $\psi(x,t)$, which solves the Schrödinger equation for a potential $V(x)$.\\
	The potential is modified such that the new potential is $V(x) + V_0$.
	In classical mechanics this addition doesn't have a meaning.
	What about quantum mechanics?
	\begin{enumerate}
		\item Show that the new solution to the Schrödinger equation has the form $\psi(x,t) e^{-\frac{i V_0 t}{h}}$.
		\item How would this change in potential change the expectation values of the position operator $\hat{x}$, the momentum operator $\hat{p}$, and a general operator $\hat{Q}\left( \hat{x},\hat{p} \right)$?
	\end{enumerate}
\end{question}

\begin{solution}
	\begin{enumerate}[leftmargin=*]
		\item
			Let
			\begin{align*}
				\Psi(x,t) & = \psi(x)
			\end{align*}
			As it a solution to the Schrödinger equation,
			\begin{align*}
				i \hbar \frac{\psi(x) \varphi'(t)}{\psi(x) \varphi(t)} & = V_0
			\end{align*}
			Therefore,
			\begin{align*}
				i \hbar \frac{\psi(x) \tilde{\varphi}'(t)}{\psi(x) \tilde{\varphi}(t)} & = V_0 + V(x)
			\end{align*}
			Therefore,
			\begin{align*}
				\tilde{\varphi(t)} & = e^{-\frac{i V(x) t}{\hbar}} e^{-\frac{i V_0 t}{\hbar}}
			\end{align*}
			Therefore,
			\begin{align*}
				\Psi(x,t) & = \psi(x) \tilde{\varphi}(t) \\
                                          & = \Psi(x,t) e^{-\frac{i V_0 t}{\hbar}}
			\end{align*}
			\qed
		\item
			\begin{align*}
				\tilde{\langle x \rangle} & = \int\limits_{-\infty}^{\infty} \tilde{\Psi}^* x \tilde{\Psi} \dif x                                      \\
                                                          & = \int\limits_{-\infty}^{\infty} \Psi^* e^{\frac{i V_0 t}{\hbar}} x \Psi e^{-\frac{i V_0 t}{\hbar}} \dif x \\
                                                          & = \int\limits_{-\infty}^{\infty} \Psi^* x \Psi \dif x                                                      \\
                                                          & = \langle x \rangle
			\end{align*}
			Therefore, the change in potential \emph{does not change} the expectation value of $x$.
			\begin{align*}
				\tilde{\langle p \rangle} & = \int\limits_{-\infty}^{\infty} \tilde{\Psi}^* \left( -i \hbar \dod{}{x} \tilde{\Psi} \right) \dif x                                      \\
                                                          & = \int\limits_{-\infty}^{\infty} \Psi^* e^{\frac{i V_0 t}{\hbar}} \left( -i \hbar \dod{}{x} \Psi e^{-\frac{i V_0 t}{\hbar}} \right) \dif x \\
                                                          & = \int\limits_{-\infty}^{\infty} \Psi^* e^{\frac{i V_0 t}{\hbar}} e^{-\frac{i V_0 t}{\hbar}} \left( -i \hbar \dod{}{x} \Psi \right) \dif x \\
                                                          & = \int\limits_{-\infty}^{\infty} \Psi^*(x,t) \left( -i \hbar \dod{}{x} \Psi(x,t) \right) \dif x                                            \\
                                                          & = \langle p \rangle
			\end{align*}
			Therefore, the change in potential \emph{does not change} the expectation value of $p$.
			\begin{align*}
				\tilde{\langle Q \rangle} & = \int\limits_{-\infty}^{\infty} \tilde{\Psi}^* \hat{Q} \tilde{\Psi} \dif x                                      \\
                                                          & = \int\limits_{-\infty}^{\infty} \Psi^* e^{\frac{i V_0 t}{\hbar}} \hat{Q} \Psi e^{-\frac{i V_0 t}{\hbar}} \dif x \\
                                                          & = \int\limits_{-\infty}^{\infty} \Psi^* \hat{Q} \Psi \dif x                                                      \\
                                                          & = \langle Q \rangle
			\end{align*}
			Therefore, the change in potential \emph{does not change} the expectation value of $Q$.
	\end{enumerate}
\end{solution}

\begin{question}
	Prove the following commutation relations.
	\begin{enumerate}
		\item $\left[ \hat{p},\hat{x}^n \right] = -i \hbar n \hat{x}^{n - 1}$
		\item $\left[ \hat{x},\hat{p}^2 \right] = 2 i \hbar \hat{p}$
	\end{enumerate}
\end{question}

\begin{question}
	\begin{enumerate}[leftmargin=*]
		\item
			\begin{align*}
				\left[ \hat{p},\hat{x}^n \right]f(x) & = \hat{p} \hat{x}^n f(x) - \hat{x}^n \hat{p} f(x)                          \\
                                                                     & = \hat{p} x^n f(x) - \hat{x}^n \left( -i \hbar \dod{}{x}f(x) \right)       \\
                                                                     & = -i \hbar \dod{}{x}\left( x^n f(x) \right) + i \hbar x^n f'(x)            \\
                                                                     & = -i \hbar \left( n x^{n - 1} f(x) + x^n f'(x) \right) + i \hbar x^n f'(x) \\
                                                                     & = -i \hbar n x^{n - 1} f(x)                                                \\
                                                                     & = -i \hbar n \hat{x}^{n - 1} f(x)
			\end{align*}
			Therefore,
			\begin{align*}
				\left[ \hat{p},\hat{x}^n \right] & = -i \hbar n \hat{x}^{n - 1}
			\end{align*}
			\qed
		\item
			\begin{align*}
				\left[ \hat{x},\hat{p}^2 \right]f(x) & = \hat{x} \hat{p}^2 f(x) - \hat{p}^2 \hat{x} f(x)                                                                               \\
                                                                     & = \hat{x} \hat{p} \hat{p} f(x) - \hat{p} \hat{p} \hat{x} f(x)                                                                   \\
                                                                     & = \hat{x} \hat{p} \left( -i \hbar \dod{}{x} f(x) \right) - \hat{p} \hat{p} x f(x)                                               \\
                                                                     & = \hat{x} \left( -i \hbar \dod{}{x}\left( \dod{}{x} f(x) \right) \right) - \hat{p} \left( -i \hbar \dod{}{x} x f(x) \right)     \\
                                                                     & = x \left( -i \hbar \dod{}{x}\left( \dod{}{x} f(x) \right) \right) + i \hbar \dod{}{x} \left( -i \hbar \dod{}{x} x f(x) \right) \\
                                                                     & = x \left( -i \hbar f''(x) \right) + i \hbar \dod{}{x} \left( -i \hbar x f'(x) - i \hbar f(x) \right)                           \\
                                                                     & = -i \hbar x f''(x) + i \hbar \left( -i \hbar x f''(x) - i \hbar f'(x) - i \hbar f'(x) \right)                                  \\
                                                                     & = i \hbar \left( -x f''(x) - i \hbar x f''(x) - i \hbar f'(x) - i \hbar f'(x)  \right)                                          \\
                                                                     & = -i \hbar \left( x f''(x) + i \hbar x f''(x) + i \hbar f'(x) + i \hbar f'(x)  \right)                                          \\
                                                                     & = 2 i \hbar \hat{p} f(x)
			\end{align*}
			Therefore,
			\begin{align*}
				\left[ \hat{x},\hat{p}^2 \right] & = 2 i \hbar \hat{p}
			\end{align*}
			\qed
	\end{enumerate}
\end{question}

\end{document}
