\documentclass[fleqn, a4paper, 11pt, oneside]{amsart}
%\usepackage[top = 2cm, bottom = 1cm, left = 1cm, right = 1cm]{geometry}
\usepackage{exsheets, tasks}
\usepackage{amsmath, amssymb, amsthm} %standard AMS packages
\usepackage{marginnote} %marginnotes
\usepackage{gensymb} %miscellaneous symbols
\usepackage{commath} %differential symbols
\usepackage{xcolor} %colours
\usepackage{cancel} %cancelling terms
\usepackage[free-standing-units, space-before-unit]{siunitx} %formatting units
	\sisetup
	{
		per-mode=fraction,
		fraction-function=\frac
	}
\usepackage{tikz, pgfplots} %diagrams
\usetikzlibrary{calc, hobby, patterns, intersections, decorations.markings}
\usepackage{graphicx} %inserting graphics
\usepackage{hyperref} %hyperlinks
\usepackage{datetime} %date and time
\usepackage{ulem} %underline for \emph{}
\usepackage{xfrac} %inline fractions
\usepackage{enumerate,enumitem} %numbered lists
\usepackage{float} %inserting floats
\usepackage{circuitikz}[american voltages, american currents] %circuit diagrams
\usepackage[utf8]{inputenc}
\usepackage{booktabs}
\usepackage{todonotes}

\newcommand\numberthis{\addtocounter{equation}{1}\tag{\theequation}} %adds numbers to specific equations in non-numbered list of equations

\newcommand{\AxisRotator}[1][rotate=0]{
	\tikz [x=0.25cm,y=0.60cm,line width=.2ex,-stealth,#1] \draw (0,0) arc (-150:150:1 and 1);%
} %rotation symbols on axes

\theoremstyle{definition}
\newtheorem{example}{Example}
\newtheorem{definition}{Definition}

\theoremstyle{theorem}
\newtheorem{theorem}{Theorem}

\newcommand{\curl}{\mathrm{curl\,}}

\makeatletter
\@addtoreset{section}{part} %resets section numbers in new part
\makeatother

\renewcommand{\thesubsection}{(\arabic{subsection})}
\renewcommand{\thesection}{(\arabic{section})}

\renewcommand{\emph}{\uline}

\renewcommand{\tilde}{\widetilde}

%section headings on left
\makeatletter
\def\specialsection{\@startsection{section}{1}%
	\z@{\linespacing\@plus\linespacing}{.5\linespacing}%
	%  {\normalfont\centering}}% DELETED
	{\normalfont}}% NEW
\def\section{\@startsection{section}{1}%
	\z@{.7\linespacing\@plus\linespacing}{.5\linespacing}%
	%  {\normalfont\scshape\centering}}% DELETED
	{\normalfont\scshape}}% NEW
\makeatother

%forces newline after subsection
\makeatletter
\def\subsection{\@startsection{subsection}{3}%
	\z@{.5\linespacing\@plus.7\linespacing}{.1\linespacing}%
	{\normalfont\itshape}}
\makeatother

\settasks{counter-format = tsk[1].}

\SetupExSheets{solution/print = true}

%opening
\title{Quantum and Solid State Physics : Assignment 13}
\author
{
	Aakash Jog\\
	ID : 989323563
}
\date{\formatdate{19}{1}{2016}}

\begin{document}

\tikzset{->-/.style={decoration={
  markings,
  mark=at position #1 with {\arrow{>}}},postaction={decorate}}}

\maketitle
%\setlength{\mathindent}{0pt}

\begin{question}
	Suppose we have a PN junction at equilibrium, which means we bring in contact a P-type material and an N-type material, with contacts at both ends attached to ground.
	The net current across the device is zero, i.e.
	\begin{align*}
		J_{\text{total}} & = 0
	\end{align*}
	Explain why we expect to have a built-in electric field in this device.
\end{question}

\begin{solution}
	As the two parts of the junction are P-type and N-type, they have excess holes and electrons, respectively.
	Therefore, due to the carrier gradients, the electrons from the N-type part, and the holes from the P-type part drift towards the respective other regions.
	Therefore the N-type part is positively charged, and the P-type part is negatively charged.
	Therefore, an electric field must be generated.
\end{solution}

\begin{question}
	At $t = 0$, consider the electron in an Hydrogen atom, described by the wave function
	\begin{align*}
		\psi & = \frac{1}{6} \left( 4 \psi_{1 0 0} + 3 \psi_{2 1 1} - \psi_{2 1 0} + \sqrt{10} \psi_{2 1 (-1)} \right)
	\end{align*}
	where $\psi_{n l m}$ are normalized eigenfunctions of the energy operator for the Coulomb potential.
	\begin{enumerate}
		\item
			Suppose we measure the energy of the electron, what energies can we measure, in \si{\electronvolt} and with what probabilities?
		\item
			Suppose we measure the energy corresponding to the lowest probability from above.
			Write $\psi(r,\theta,\varphi,t)$ after the measurement, in the form $\psi_{n l m}$.
			Is the electron in a stationary state?
	\end{enumerate}
	Consider the original electron described by
	\begin{align*}
		\psi & = \frac{1}{6} \left( 4 \psi_{1 0 0} + 3 \psi_{2 1 1} - \psi_{2 1 0} + \sqrt{10} \psi_{2 1 (-1)} \right)
	\end{align*}
	\begin{enumerate}[resume]
		\item
			Suppose we measure the total angular momentum $L^2$ of the electron, what values can we measure, in terms of $\hbar$, and with what probabilities?
		\item
			Suppose we measure the total angular momentum $L^2$ of the electron, and the result is $2 \hbar^2$.
			Write down $\psi(r,\theta,\varphi)$ after the measurement, in the form $\psi_{n l m}$.
			Hint: You should have three unknown parameters in your answer.
		\item
			Suppose that now we measure the angular momentum in the $x$-direction, $L_x$, of the electron above, right after the measurement done above.
			The result is $\hbar$.
			Find the three unknown parameters above.
			In your solution, assume that $\psi(r,\theta,\varphi)$ you found above is an eigenstate of $L_x$.
	\end{enumerate}
\end{question}

\begin{solution}
	\begin{enumerate}[leftmargin=*]
		\item
			The energy level corresponding to $\psi_{1 0 0}$ is
			\begin{align*}
				E_1 & = -13.6 \si{\electronvolt}
			\end{align*}
			The energy level corresponding to $\psi_{2 1 1}$ is
			\begin{align*}
				E_2 & = -\frac{13.6}{4} \si{\electronvolt}
			\end{align*}
			The energy level corresponding to $\psi_{2 1 0}$ is
			\begin{align*}
				E_2 & = -\frac{13.6}{4} \si{\electronvolt}
			\end{align*}
			The energy level corresponding to $\psi_{2 1 (-1)}$ is
			\begin{align*}
				E_2 & = -\frac{13.6}{4} \si{\electronvolt}
			\end{align*}
			Therefore, the probability of the measured energy being $E_1$ is
			\begin{align*}
				P(E_1) & = \left( \frac{4}{6} \right)^2 \\
                                       & = \frac{16}{36}                \\
                                       & = \frac{4}{9}
			\end{align*}
			Therefore, the probability of the measured energy being $E_2$ is
			\begin{align*}
				P(E_2) & = \left( \frac{3}{6} \right)^2 + \left( \frac{-1}{6} \right)^2 + \left( \frac{\sqrt{10}}{6} \right)^2 \\
                                       & = \frac{5}{9}
			\end{align*}
		\item
			If the energy corresponding to the lowest probability, i.e. $E_1$ is measured,
			\begin{align*}
				\psi(r,\theta,\varphi,t) & = \psi(r,\theta,\varphi,0) e^{-\frac{i E_1 t}{\hbar}} \\
                                                         & = \frac{2}{3} \psi_{1 0 0} e^{-\frac{i E_1 t}{\hbar}}
			\end{align*}
			Therefore, the electron is in a stationary state.
		\item
			The value of $L^2$ corresponding to $\psi_{1 0 0}$ is
			\begin{align*}
				l (l + 1) \hbar^2 & = 0
			\end{align*}
			The value of $L^2$ corresponding to $\psi_{2 1 1}$ is
			\begin{align*}
				l (l + 1) \hbar^2 & = 2 \hbar^2
			\end{align*}
			The value of $L^2$ corresponding to $\psi_{2 1 0}$ is
			\begin{align*}
				l (l + 1) \hbar^2 & = 2 \hbar^2
			\end{align*}
			The value of $L^2$ corresponding to $\psi_{2 1 (-1)}$ is
			\begin{align*}
				l (l + 1) \hbar^2 & = 2 \hbar^2
			\end{align*}
			Therefore, the probability of the measured value of $L^2$ being $0$ is
			\begin{align*}
				P(0) & = \left( \frac{4}{6} \right)^2 \\
                                     & = \frac{4}{9}
			\end{align*}
			Therefore, the probability of the measured value of $L^2$ being $2 \hbar^2$ is
			\begin{align*}
				P\left( 2 \hbar^2 \right) & = \left( \frac{3}{6} \right)^2 + \left( \frac{-1}{6} \right)^2 + \left( \frac{\sqrt{10}}{6} \right)^2 \\
                                                          & = \frac{5}{9}
			\end{align*}
		\item
			\begin{align*}
				\psi(r,\theta,\varphi) & = \frac{1}{6} \left( A \psi_{2 1 1} + B \psi_{2 1 0} + C \psi_{2 1 (-1)} \right)
			\end{align*}
		\item
			\begin{align*}
				\hat{L}_x & = \frac{\hat{L}_+ + \hat{L}_-}{2}
			\end{align*}
			Therefore,
			\begin{align*}
				\hat{L}_x \psi & = \frac{1}{2} \hat{L}_+ \psi + \frac{1}{2} \hat{L}_- \psi                                                   \\
                                               & = \quad \frac{1}{2} \frac{1}{6} \hat{L}_+\left( A \psi_{2 1 1} + B \psi_{2 1 0} + C \psi_{2 1 (-1)} \right) \\
                                               & \quad + \frac{1}{2} \frac{1}{6} \hat{L}_-\left( A \psi_{2 1 1} + B \psi_{2 1 0} + C \psi_{2 1 (-1)} \right) \\
                                               & = \quad \frac{1}{12} \left( A \psi_{2 2 1} + B \psi_{2 2 0} + C \psi_{2 2 (-1)} \right)                     \\
                                               & \quad + \frac{1}{12} \left( A \psi_{2 0 1} + B \psi_{2 0 0} + C \psi_{2 0 (-1)} \right)                     \\
                                               & = \frac{A}{12} \left( \psi_{2 2 1} + \psi_{2 0 1} \right) + \frac{B}{12} \left( \psi_{2 2 0} + \psi_{2 0 0} \right) + \frac{C}{12} \left( \psi_{2 2 (-1)} + \psi_{2 0 (-1)} \right)
			\end{align*}
	\end{enumerate}
\end{solution}

\end{document}
