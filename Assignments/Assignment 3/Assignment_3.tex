\documentclass[fleqn, a4paper, 11pt, oneside]{amsart}
%\usepackage[top = 2cm, bottom = 1cm, left = 1cm, right = 1cm]{geometry}
\usepackage{exsheets, tasks}
\usepackage{amsmath, amssymb, amsthm} %standard AMS packages
\usepackage{marginnote} %marginnotes
\usepackage{gensymb} %miscellaneous symbols
\usepackage{commath} %differential symbols
\usepackage{xcolor} %colours
\usepackage{cancel} %cancelling terms
\usepackage[free-standing-units, space-before-unit]{siunitx} %formatting units
\usepackage{tikz, pgfplots} %diagrams
\usetikzlibrary{calc, hobby, patterns, intersections, decorations.markings}
\usepackage{graphicx} %inserting graphics
\usepackage{hyperref} %hyperlinks
\usepackage{datetime} %date and time
\usepackage{ulem} %underline for \emph{}
\usepackage{xfrac} %inline fractions
\usepackage{enumerate,enumitem} %numbered lists
\usepackage{float} %inserting floats
\usepackage{circuitikz}[american voltages, american currents] %circuit diagrams
\usepackage[utf8]{inputenc}

\newcommand\numberthis{\addtocounter{equation}{1}\tag{\theequation}} %adds numbers to specific equations in non-numbered list of equations

\newcommand{\AxisRotator}[1][rotate=0]{
	\tikz [x=0.25cm,y=0.60cm,line width=.2ex,-stealth,#1] \draw (0,0) arc (-150:150:1 and 1);%
} %rotation symbols on axes

\theoremstyle{definition}
\newtheorem{example}{Example}
\newtheorem{definition}{Definition}

\theoremstyle{theorem}
\newtheorem{theorem}{Theorem}

\newcommand{\curl}{\mathrm{curl\,}}

\makeatletter
\@addtoreset{section}{part} %resets section numbers in new part
\makeatother

\renewcommand{\thesubsection}{(\arabic{subsection})}
\renewcommand{\thesection}{(\arabic{section})}

\renewcommand{\emph}{\uline}

%section headings on left
\makeatletter
\def\specialsection{\@startsection{section}{1}%
	\z@{\linespacing\@plus\linespacing}{.5\linespacing}%
	%  {\normalfont\centering}}% DELETED
	{\normalfont}}% NEW
\def\section{\@startsection{section}{1}%
	\z@{.7\linespacing\@plus\linespacing}{.5\linespacing}%
	%  {\normalfont\scshape\centering}}% DELETED
	{\normalfont\scshape}}% NEW
\makeatother

%forces newline after subsection
\makeatletter
\def\subsection{\@startsection{subsection}{3}%
	\z@{.5\linespacing\@plus.7\linespacing}{.1\linespacing}%
	{\normalfont\itshape}}
\makeatother

\settasks{counter-format = tsk[1].}

\SetupExSheets{solution/print = true}

%opening
\title{Quantum and Solid State Physics : Assignment 2}
\author
{
	Aakash Jog\\
	ID : 989323563
}
\date{\formatdate{5}{11}{2015}}

\begin{document}

\tikzset{->-/.style={decoration={
  markings,
  mark=at position #1 with {\arrow{>}}},postaction={decorate}}}

\maketitle
%\setlength{\mathindent}{0pt}

\begin{question}
	If we require a silicon sample to have an electron concentration 4 times higher than its hole concentration, i.e. $n = 4 p$, which of the following dopant concentration would you choose, expressed in terms of intrinsic carrier concentration $n_i$?
	\begin{enumerate}
		\item $N_D = 0.5 n_i$
		\item $N_D = 2 n_i$ \label{correct_answer_1}
		\item $N_D = 1.5 n_i$
		\item $N_A = 0.5 n_i$
		\item $N_A = 2 n_i$
		\item $N_A = 1.5 n_i$
	\end{enumerate}
\end{question}

\begin{solution}
	As the requirement is to have an electron concentration higher than the original sample, the dopant must be a donor.\\
	For the original sample,
	\begin{align*}
		n p &= {n_i}^2\\
	\end{align*}
	Therefore, for the doped sample,
	\begin{align*}
		4 p n &= {N_D}^2\\
		\therefore 4 {n_i}^2 &= {N_D}^2
	\end{align*}
	Therefore,
	\begin{align*}
		N_D &= 2 n_i
	\end{align*}
	Hence, the dopant concentration should be \emph{(\ref{correct_answer_1}) $N_D = 2 n_i$}.
\end{solution}

\begin{question}
	Let's examine 3 wafers, all of which are doped with a concentration of $10^{15}$ dopants \si{\per\cubic\centi\metre}.
	\begin{enumerate}
		\item Wafer A: Silicon, in which the dopant energy level is 1.095 \electronvolt above the valence energy level $E_V$.
		\item Wafer B: Silicon, in which the dopant energy level is 1.08 \electronvolt above the valence energy level $E_V$.
		\item Wafer C: GaAs, in which the dopant energy level is 1.095 \electronvolt above the valence energy level $E_V$.
	\end{enumerate}
	\begin{enumerate}
		\item
			Do we expect the dopants in wafers A, B, and C to be donors or acceptors?
			Draw the energy diagrams for each wafer, and clearly label all relevant energy levels and spacing between them.
		\item
			As the temperature is increased, which wafer will be the first to transition from the ionization region to the extrinsic region?
		\item
			For which wafer will intrinsic carriers start to dominate at the lowest temperature?
	\end{enumerate}
\end{question}

\begin{solution}
	\begin{enumerate}[leftmargin=*]
		\item
			As the dopant energy in the wafers is closer to $E_C$, than $E_V$, the dopants will be \emph{donors}.
			\begin{figure}[H]
				\centering
				\begin{tikzpicture}
					\def\l{4};
					\def\eValence{1};
					\def\eConduction{4};
					\def\eDonor{3.5};
			
					\newcommand{\drawEnergyLevel}[1]
					{
						\draw (0,#1) -- (\l,#1)
					}
			
					\newcommand{\drawEnergyGap}[4]
					{
						\draw [|<->|] (#3,#1) -- (#3,#2) node [midway, left] {#4}
					}
			
					\begin{scope}
						\drawEnergyLevel{\eValence} node [right] {$E_V$};
						\drawEnergyLevel{\eConduction} node [above right] {$E_C$};
						\drawEnergyLevel{\eDonor} node [below right] {$E_D$};
					\end{scope}
			
					\begin{scope}
						\drawEnergyGap{\eValence}{\eConduction}{-3}{$E_{\textnormal{gap}} = 1.1 \electronvolt$};
						\drawEnergyGap{\eValence}{\eDonor}{-1}{$1.095 \electronvolt$};
					\end{scope}
				\end{tikzpicture}
				\caption{Energy Diagram: Wafer A}
			\end{figure}

			\begin{figure}[H]
				\centering
				\begin{tikzpicture}
					\def\l{4};
					\def\eValence{1};
					\def\eConduction{4};
					\def\eDonor{3.5};
			
					\newcommand{\drawEnergyLevel}[1]
					{
						\draw (0,#1) -- (\l,#1)
					}
			
					\newcommand{\drawEnergyGap}[4]
					{
						\draw [|<->|] (#3,#1) -- (#3,#2) node [midway, left] {#4}
					}
			
					\begin{scope}
						\drawEnergyLevel{\eValence} node [right] {$E_V$};
						\drawEnergyLevel{\eConduction} node [above right] {$E_C$};
						\drawEnergyLevel{\eDonor} node [below right] {$E_D$};
					\end{scope}
			
					\begin{scope}
						\drawEnergyGap{\eValence}{\eConduction}{-3}{$E_{\textnormal{gap}} = 1.1 \electronvolt$};
						\drawEnergyGap{\eValence}{\eDonor}{-1}{$1.08 \electronvolt$};
					\end{scope}
				\end{tikzpicture}
				\caption{Energy Diagram: Wafer B}
			\end{figure}

			\begin{figure}[H]
				\centering
				\begin{tikzpicture}
					\def\l{4};
					\def\eValence{1};
					\def\eConduction{4};
					\def\eDonor{3.5};
			
					\newcommand{\drawEnergyLevel}[1]
					{
						\draw (0,#1) -- (\l,#1)
					}
			
					\newcommand{\drawEnergyGap}[4]
					{
						\draw [|<->|] (#3,#1) -- (#3,#2) node [midway, left] {#4}
					}
			
					\begin{scope}
						\drawEnergyLevel{\eValence} node [right] {$E_V$};
						\drawEnergyLevel{\eConduction} node [above right] {$E_C$};
						\drawEnergyLevel{\eDonor} node [below right] {$E_D$};
					\end{scope}
			
					\begin{scope}
						\drawEnergyGap{\eValence}{\eConduction}{-3}{$E_{\textnormal{gap}} = 1.43 \electronvolt$};
						\drawEnergyGap{\eValence}{\eDonor}{-1}{$1.095 \electronvolt$};
					\end{scope}
				\end{tikzpicture}
				\caption{Energy Diagram: Wafer C}
			\end{figure}
		\item
			When the temperature is increased, the wafer for which the gap between $E_D$ and $E_C$ is the least will be the first to transition into the extrinsic region.
			Therefore, \emph{wafer A} will be the first to transition.
		\item
			The energy band gap of Si is less than that of GaAs.
			However, in wafer B, the dopant charge carriers are at a lower energy level than that of those in wafer A.
			Hence, the intrinsic carriers in \emph{wafer B} will start to dominate at the lowest temperature.
	\end{enumerate}
\end{solution}

\begin{question}
	Two wafers of silicon are doped with the following concentration
	\begin{align*}
		{N_D}_1 &= 10^{15} \si{\per\cubic\centi\metre}\\
		{N_D}_2 &= 10^{16} \si{\per\cubic\centi\metre}
	\end{align*}
	Is it possible that the electron concentration $n$ in both wafers is identical, i.e. $n_1 = n_2$?
	\begin{enumerate}
		\item No
		\item Yes, in the extrinsic region
		\item Yes, in the intrinsic region
		\item Yes, in both the extrinsic and intrinsic regions
	\end{enumerate}
\end{question}

\begin{solution}
	\begin{align*}
		n &= N_D + \text{number of thermally generated EHPs}
	\end{align*}
	Therefore,
	\begin{align*}
		n_1 &= {N_D}_1 + \text{number of thermally generated EHPs}\\
		n_2 &= {N_D}_2 + \text{number of thermally generated EHPs}
	\end{align*}
	Therefore,
	\begin{align*}
		n_1 &\neq n_2
	\end{align*}
	Therefore, it is \emph{not possible} for the electron concentration in both wafers to be identical.
\end{solution}

\begin{question}
	A particle is represented, at time $t = 0$, by the following wave function.
	\begin{align*}
		\psi(x,0) &=
			\begin{cases}
				A \frac{x}{a} &;\quad 0 \le x \le a\\
				A \frac{b - x}{b - a} &;\quad a \le x \le b\\
				0 &;\quad \text{otherwise}\\
			\end{cases}
	\end{align*}
	\begin{enumerate}
		\item Determine the normalization constant $A$.
		\item Sketch $\psi(x,0)$ as a function of $x$.
		\item Where is the particle most likely to be found at $t = 0$?
		\item What is the probability of finding the particle to the left of $a$?
		\item What is the expectation value of $x$, at time $t = 0$?
	\end{enumerate}
\end{question}

\begin{solution}
	\begin{enumerate}[leftmargin=*]
		\item
			\begin{align*}
				1 & = \int\limits_{-\infty}^{\infty} \left| \psi(x,0) \right|^2 \dif x                                                        \\
                                  & = \int\limits_{0}^{a} A^2 \frac{x^2}{a^2} \dif x + \int\limits_{a}^{b} A^2 \frac{(b - x)^2}{(b - a)^2} \dif x                             \\
				  &= \frac{A^2}{a^2} \left. \frac{x^3}{3} \right|_{0}^{a} + \frac{A^2}{(b - a)^2} \left. b^2 x - b x^2 + \frac{x^3}{3} \right|_{a}^{b}\\
				  &= \frac{A^2}{a^2} \frac{a^3}{3} + \frac{A^2}{(b - a)^2} \left( b^3 - b^3 + \frac{b^3}{3} - a b^2 + a^2 b - \frac{a^3}{3} \right)\\
				  &= \frac{A^2 a}{3} + \frac{A^2}{(b - a)^2} \left( \frac{b^3}{3} - a b^2 + a^2 b - \frac{a^3}{3} \right)\\
				  &= A^2 \left( \frac{a}{3} + \frac{b^3}{3 (b - a)^2} - \frac{a b^2}{(b - a)^2} + \frac{a^2 b}{(b - a)^2} - \frac{a^3}{3 (b - a)^2} \right)\\
				  &= A^2 \left( \frac{a (b - a)^2 + b^3 - 3 a b^2 + 3 a^2 b - a^3}{3 (b - a)^2} \right)\\
				  &= A^2 \left( \frac{b (b - a)^2}{3 (b - a)^2} \right)\\
				  &= \frac{A^2 b}{3}
			\end{align*}
			Therefore,
			\begin{align*}
				A &= \sqrt{\frac{3}{b}}
			\end{align*}
		\item
			\begin{align*}
				A &= \sqrt{\frac{3}{b}}
			\end{align*}
			Therefore,
			\begin{align*}
				\psi(x,0) &=
					\begin{cases}
						\sqrt{\frac{3}{b}} \frac{x}{a} &;\quad 0 \le x \le a\\
						\sqrt{\frac{3}{b}} \frac{b - x}{b - a} &;\quad a \le x \le b\\
						0 &;\quad \text{otherwise}\\
					\end{cases}
			\end{align*}
			Therefore,
			\begin{figure}[H]
				\centering
				\begin{tikzpicture}
					\def\xMIN{0};
					\def\xMAX{5};
					\def\yMIN{0};
					\def\yMAX{3};
					
					\def\a{2};
					\def\b{3};
					\def\A{1};

					\begin{scope}[-stealth]
						\draw (\xMIN,0) -- (\xMAX,0) node [right] {$x$};
						\draw (0,\yMIN) -- (0,\yMAX) node [above] {$\psi(x,0)$};
					\end{scope}

					\begin{scope}
						\filldraw (\a,0) circle (1pt) node [below] {$a$};
						\filldraw (\b,0) circle (1pt) node [below] {$b$};
						\filldraw (0,\A) circle (1pt) node [left] {$A$};
					\end{scope}

					\begin{scope}
						\draw (0,0) -- (\a,\A) -- (\b,0);
					\end{scope}
				\end{tikzpicture}
			\end{figure}
		\item
			As the value of $|\psi(x)|^2$ is maximum at $x = a$, the particle is most likely to be found at \emph{$x = a$}.
		\item
			The probability of finding the particle at to the left of $a$ is
			\begin{align*}
				\int\limits_{0}^{a} \left| \psi(x,0) \right|^2 \dif x\\
				&= \int\limits_{0}^{a} \frac{3 x^2}{b a^2} \dif x\\
				&= \left. \frac{x^3}{a^2 b} \right|_{0}^{a}\\
				&= \frac{a}{b}
			\end{align*}
		\item
			\begin{align*}
				\langle x \rangle &= \int\limits_{-\infty}^{\infty} x \left| \psi(x) \right|^2 \dif x\\
				&= \int\limits_{0}^{a} x \left( \frac{A^2 x^2}{a^2} \right) \dif x + \int\limits_{a}^{b} x \frac{A^2 (b - x)^2}{(b - a)^2} \dif x\\
				&= \frac{1}{4 b} \left( 6 a^2 - 2 a b - b^2 \right)
			\end{align*}
	\end{enumerate}
\end{solution}

\begin{question}
	A particle is represented, at time $t = 0$, by the following wave function.
	\begin{align*}
		\psi(x,0) &=
			\begin{cases}
				A \left( a^2 - x^2 \right) &;\quad -a \le x \le a\\
				0 &;\quad \text{otherwise}\\
			\end{cases}
	\end{align*}
	\begin{enumerate}
		\item
			Determine the normalization constant $A$.
		\item
			What is the expectation value of $x$, at time $t = 0$?
		\item
			What is the expectation value of $p$, at time $t = 0$?
			Can you answer this part based on Ehrenfest's theorem?
			Explain your answer
		\item
			What is the expectation value of $x^2$, at time $t = 0$?
		\item
			What is the expectation value of $p^2$, at time $t = 0$?
	\end{enumerate}
\end{question}

\begin{solution}
	\begin{enumerate}[leftmargin=*]
		\item
			\begin{align*}
				1 & = \int\limits_{-\infty}^{\infty} \left| \psi(x,0) \right|^2 \dif x\\
				&=\int\limits_{-a}^{q} A^2 \left( a^2 - x^2 \right) \dif x\\
				&= \frac{16 a^2}{15 A^2}
			\end{align*}
			Therefore,
			\begin{align*}
				A^2 &= \frac{15}{16 a^2}
			\end{align*}
			Therefore,
			\begin{align*}
				A &= \sqrt{\frac{15}{16 a^2}}
			\end{align*}
		\item
			\begin{align*}
				\langle x \rangle &= \int\limits_{-a}^{a} x \left| \psi(x,0) \right|^2 \dif x\\
				&= \int\limits_{-a}^{a} x A^2 \left( a^2 - x^2 \right)^2 \dif x
			\end{align*}
			As the function is odd, and the interval is symmetric across the origin, the integral is zero.\\
			Therefore,
			\begin{align*}
				\langle x \rangle &= 0
			\end{align*}
		\item
			By Ehrenfest's theorem,
			\begin{align*}
				\langle p \rangle &= m \langle v \rangle\\
				&= m \dod{\langle x \rangle}{t}\\
				&= 0
			\end{align*}
		\item
			\begin{align*}
				\left\langle x^2 \right\rangle &= \int\limits_{-a}^{a} x^2 \left| \psi(x,0) \right|^2 \dif x\\
				&= \int\limits_{-a}^{a} x^2 A^2 \left( a^2 - x^2 \right)^2 \dif x\\
				&= 2 \int\limits_{0}^{a} x^2 A^2 \left( a^2 - x^2 \right)^2 \dif x\\
				&= \frac{a^2}{7}
			\end{align*}
		\item
			\begin{align*}
				\left\langle p^2 \right\rangle &= \int\limits_{-a}^{a} \psi^*(x,0) \left( -\hbar^2 \right) \dpd[2]{}{x} \psi(x,0) \dif x\\
				&= \int\limits_{-a}^{a} A \left( a^2 - x^2 \right) \left( -\hbar^2 \right) \dpd[2]{}{x} A \left( a^2 - x^2 \right) \dif x\\
				&= \frac{5 \hbar^2}{2 a^2}
			\end{align*}
	\end{enumerate}
\end{solution}

\end{document}
